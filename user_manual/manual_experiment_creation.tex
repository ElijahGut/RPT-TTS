% !TEX root = LMEDS_manual.tex

%%%%%%%%%%%%%%%%%%%%%
\section{Creating your own experiments}
%%%%%%%%%%%%%%%%%%%%%

%%%%%%%%%%%%%%%%%%%%%
\subsection{User-defined test components}
%%%%%%%%%%%%%%%%%%%%%

An LMEDS test has the following components

\subsubsection{Audio files}

\paragraph{}
An audio file can be any common type (.mp3, .wav, etc.).   Some web browsers are unable to play audio in certain file types.  Personally, I encode every file as both .mp3 and .ogg

\paragraph{}
See this page for more information on browser compatibility:

\url{http://www.w3schools.com/tags/tag\_audio.asp}

\paragraph{}
Audacity makes it very easy to convert entire directories of audio files into multiple audio formats with it's `chain' functionality.  

\url{http://audacityteam.org/}

\paragraph{}
To save server space and speed up loading times for users with slower internet, I reencode my wav files to 16,000 HZ before I convert them to .mp3 and .ogg.  This step is not required.  Sox is a freely available command line tool that can do this task easily:

\begin{lstlisting}
sox input_file_name -r 16000 output_file_name rate -v 96k
\end{lstlisting}

\url{http://sox.sourceforge.net/}

\subsubsection{Transcription files}

\paragraph{}
Only raw txt files with the extension .txt are accepted.  Each transcription file should contain the transcript for one stimulus.

\paragraph{}
For long excerpts, you should specify the line breaks explicitly by chunking the text into pieces over the span of several lines.  For a long excerpt, if all of the text lies on one line, the text will run off the page in LMEDS.

\subsubsection{Sequence file (see section \ref{sec:sequenceSpec})}

\paragraph{}
The file that specifies the control flow of a test.  It specifies what instructions and stimuli users will be presented with in a test and in what order. 

\paragraph{}
Multiple sequence files can exist for a set of audio and transcription files (e.g. one could have a sequence with the stimuli in a certain order and another sequence with the stimuli in a different order or with different instructions, etc.).

\subsubsection{Dictionary file(s) (see section \ref{sec:dictionarySpec})}

\paragraph{}
The file that contains all text that will be seen by users in your experiment.  This file is independent of a sequence file so one could write a sequence and present it in both English and Russian, for example.

\subsubsection{Survey file(s) (see section \ref{sec:surveySpec})}

\paragraph{}
A survey file can be used to get various kinds of feedback, such as might be needed for a longform survey or a short-answer question.  Supports text boxes, checkboxes, and radio buttons.

\paragraph{}
Each survey requires its own survey file (e.g. presurvey.txt, postsurvey.txt).

\paragraph{}
Most pages have a static layout (defined in the python code) that is filled in with the same type of content.  A survey is different because the amount of content and inputs vary with each survey page, whereas all AXB pages, for example, share the same number of inputs and outputs.  For this reason, each survey requires its own specification file while other types of pages do not.

\subsubsection{.CGI file}

\paragraph{}
The file that defines a single experiment--via a sequence file and a language file.  The name given to this file (e.g. lmeds\_demo.cgi) is the name that participants would use to access the experiment in the URL.

\paragraph{}
A template of a typical .cgi file can be found in section \ref{sec:cgitemplate}.  An example file is also distributed in the /cgi folder included with LMEDS.

%%%%%%%%%%%%%%%%%%%%%
\subsection{Dictionary file specification}
\label{sec:dictionarySpec}
%%%%%%%%%%%%%%%%%%%%%

\paragraph{}
LMEDS uses \textbf{dictionaries} for multi-lingual support.  These dictionaries contain a set of keys where each key has text associated with it.  One dictionary might have text only in English, while another only in French, but the keys are the same in both.  Thus, when a key is used in a page, LMEDS looks in the dictionary (that is appropriate for the current experiment) for the corresponding text for that key.  E.g. the key ``dictionary'' will fetch the text ``Dictionary'' for English and  ``Dictionnaire'' for French.  Each dictionary file is created by the user.

\paragraph{}
Some keys are defined within the python code.  Other keys are user-defined.  The user\_script \textbf{generate\_language\_dictionary.py} can be used to generate a template dictionary file (all keys are present but there are no text values) or update an existing dictionary file with new keys.

\paragraph{}
You might find it useful to create a new dictionary by using another one--either if you're starting a new experiment or translating your current experiment into another language.

\paragraph{}
A dictionary files has the following components (a very small, but complete dictionary file follows these, alternatively, you can look at the included dictionary file for a real example: \textbf{/tests/lmeds\_demo/english.txt}):

\subsubsection{Sections}

\paragraph{}
A section is denoted with ``-'' characters on the line above and below the section name.  The lengths of these lines (the number of ``-'') doesn't matter but should be kept the same length to be visually more coherent.  

\paragraph{}
Sections are actually ignored in the code.  Their only purpose is to help structure dictionaries, perhaps most naturally by page type.  Keys shared by multiple pages are kept in their own sections.

\subsubsection{Keys}

\paragraph{}
A key is denoted in the same way with ``='' characters.  Keys must be the same, regardless of the language. 

\subsubsection{Texts}

\paragraph{}
A ``text'' appears after a key and before another key or section.  Unlike sections and keys, text is undecorated text.  

\paragraph{}
Text is rendered in HTML.  This means HTML markup is allowed.  It also means that whitespace other than a single space is ignored, although you may use extra whitespace to make it easier to read the text in the dictionary.  If you want a line break, for example, you will need to insert one used \texttt{<br />}

\subsubsection{Dictionary example file}

\paragraph{}
Here is a short, concrete example of a dictionary file

\paragraph{}

\begin{tcolorbox}[breakable,colback=white,colframe=blue,width=\dimexpr\textwidth+12mm\relax,enlarge left by=-6mm]

\begin{lstlisting}
---------------
text_page
---------------

===============
title
===============

Perception of Spoken Discourse

===============
first_block_finished
===============

You have completed the first block.<br /><br />

Please feel free to rest for a minute before continuing.

\end{lstlisting}

\end{tcolorbox}

\paragraph{}
The section denotes that these keys are used in the text pages, which tend to be used for giving instructions.  Whenever LMEDS needs to display the title, it will look at the text for the key ``title'' in this case ``Perception of Spoken Discourse''.

%%%%%%%%%%%%%%%%%%%%%
\subsection{Survey file specification}
\label{sec:surveySpec}
%%%%%%%%%%%%%%%%%%%%%

\paragraph{}
The survey format allows for the simple creation of short or long questionnaires.  The format is simple--the first line in a survey question contains the text prompt, subsequent lines specify one or more data entry fields that users can use to answer the question, and, finally, a blank line signals that the current item is complete. An small example survey file can be found is at the bottom of this section, two surveys are also bundled in lmeds \textbf{/tests/lmeds\_demo/presurvey.txt} and \textbf{/tests/lmeds\_demo/postsurvey.txt}.

\paragraph{}
A note on arguments: For data entry fields that have arguments, arguments should be separated from the data entry field name by a space and from each other by a comma 

e.g.

\begin{lstlisting}
Choice English, Arabic, Other
\end{lstlisting}

\paragraph{}
Here is the list of data entry fields available:

\subsubsection{None}

\paragraph{}
There are no inputs the user can choose from.  This is used to include instructions on the page.

e.g.

\begin{lstlisting}
Please answer the questions below.
None
\end{lstlisting}

\subsubsection{Choice}

\paragraph{}
Users can select exactly one item out of many.

e.g.

\begin{lstlisting}
Sex:
Choice Male, Female
\end{lstlisting}

\subsubsection{Item\_List}

\paragraph{}
Users can select as many items as they want out of many.

e.g.

\begin{lstlisting}
Indicate the language(s) that you are familiar with.
Item_List English, French, Spanish, Italian, German
\end{lstlisting}

\subsubsection{Choicebox}

\paragraph{}
A dropdown box.  Users can select one item from many.  Similar to a \textbf{Choice} but is more efficient in space for large lists of items.

e.g.

\begin{lstlisting}
Level of education completed:
Choicebox High School, Some College, Bachelor's Degree
\end{lstlisting}

\subsubsection{Textbox}

\paragraph{}
A single line for users to enter a small amount of information.  A textbox takes no arguments.

e.g.

\begin{lstlisting}
Occupation
Textbox
\end{lstlisting}

\subsubsection{Multiline\_Textbox}

\paragraph{}
A longer-form textbox that can span several lines.  It takes exactly two arguments: the number of characters across and the number of lines.

e.g.

\begin{lstlisting}
What did you think of this test?  Please provide any feedback.
Multiline_Textbox 50, 7
\end{lstlisting}

\subsubsection{Sublists}

\paragraph{}
It is possible to designate a group of items as subquestions of the previous question.  These items will be displayed tabbed.  To do this, encapsulate the relevant items in the sublist tag:

\begin{lstlisting}
<sublist>
Occupation
Textbox

City of birth
Textbox
</sublist>
\end{lstlisting}

\subsubsection{Survey file example}

\paragraph{}
Here is a short concrete example demonstrating the key features discussed above

\paragraph{}

\begin{tcolorbox}[breakable,colback=white,colframe=blue,width=\dimexpr\textwidth+12mm\relax,enlarge left by=-6mm,enlarge right by=6mm]

\begin{lstlisting}
Please answer the questions below.
None

Sex: 
Choice Male, Female

Age:
Textbox

Country of birth:
Choice United States, Other
Textbox

<sublist>
If United States, list city/state:
Textbox

If other, how old were you when you moved to the USA?
Textbox
</sublist>

\end{lstlisting}

\end{tcolorbox}

\paragraph{}
Thus we have a survey with (ignoring the first item, which doesn't take user input) 5 items in total (2 being subitems).  The third item shows how multiple data entry fields can be placed on a single question (if the user selects ``other'' for the first question they are expected to specify what they meant in the Textbox).

%%%%%%%%%%%%%%%%%%%%%
\subsection{Sequence file specification}
\label{sec:sequenceSpec}
%%%%%%%%%%%%%%%%%%%%%

\paragraph{}
The sequence file contains the items that will be presented in a test.  The first line in a test is the test name which should be prefixed with a ``*'' (e.g. *My\_Test\_Sequence) (\textbf{data from the experiment will be output to a folder with the test name.  Giving multiple sequence files the same name, means that they will dump their output to the same folder}).  The second item must be ``login'' where users create a name to associate their data with.  If login is not the item on the second line, unexpected behavior can result.
Subsequent items are presented in a linear fashion as users progress through the sequence.  The last item in the sequence must be ``end''.

\paragraph{}
\textbf{Most pages take arguments.}  These arguments specify the behavior the page should take (which audio files to play, instructions to present, etc.)  Arguments are separated from one another and from the page type by a space.  Audio files and textfiles included in an argument should not include their file extension (.txt, .wav, etc.)--LMEDS will determine the appropriate extension to be used.

\begin{lstlisting}
Prominence water water 1 3 true
\end{lstlisting}

\paragraph{}
The above example is the entry for the prominence page.  It comes with 5 arguments.  Sections \ref{sec:sequenceSpecBasic} and \ref{sec:sequenceSpecSpecific} include information on how to understand the individual arguments to a page.  A full example sequence file can be found in section \ref{sec:sequenceFileExample}.


\subsubsection{A Note on Default Values}

\paragraph{}
\textit{This is a completely optional feature.  If you find it confusing you can skip it.}

\paragraph{}
Some page types, such as \textbf{prominence} have default values for some arguments.   An argument might have a default value if alternative values are uncommon or to support new functionality without requiring changes to pre-existing code.

\paragraph{}
Arguments that have default values do not have to be specified in a sequence file if you are ok with the defaults--this can make your sequence file cleaner and easier to read and maintain.

\paragraph{}
There are two ways that you can specify an optional value.  One is to list it as normal e.g.

\begin{lstlisting}
prominence water water 1 3 acoustics true
\end{lstlisting}

\paragraph{}
A second way is to refer to it explicitly by its name as described in sections \ref{sec:sequenceSpecBasic} and \ref{sec:sequenceSpecSpecific}. e.g.

\begin{lstlisting}
prominence water 1 3 instructions=acoustics presentAudio=true
\end{lstlisting}

\paragraph{}
This second way is useful, for example, if you wanted to specify the value for a later variable such as ``presentAudio'' but not ``instructions''.  The only way to do this is by naming the variable 

\begin{lstlisting}
Prominence water 1 3 presentAudio=true
\end{lstlisting}

\paragraph{}
If one attempted the same list of arguments but without referring to the variable name, this would set ``instructions=true'' which would cause an error unless instructions named ``true'' existed in the dictionary file.

\begin{lstlisting}
Prominence water 1 3 true
\end{lstlisting}


\subsubsection{Sequence file example}
\label{sec:sequenceFileExample}

\paragraph{}
An example sequence file is also distributed with the lmeds source:

\textbf{/tests/lmeds\_demo/sequence.txt}

\paragraph{}

\begin{tcolorbox}[breakable,colback=white,colframe=blue,width=\dimexpr\textwidth+12mm\relax,enlarge left by=-6mm,enlarge right by=6mm]

\begin{lstlisting}
*LMEDS_Demo
login

text_page demo_instructions
consent demo_consent
survey presurvey

audio_list 1 1 1 [water apples water]
same_different_stream 0.5 1 -1 [water apples]
boundary_and_prominence water water 0 -1 nonspecific true
boundary_and_prominence apples apples 2 2 nonspecific true
end
\end{lstlisting}
\end{tcolorbox}

%%%%%%%%%%%%%%%%%%%%%
\subsection{Sequence file specification - common pages}
\label{sec:sequenceSpecBasic}
%%%%%%%%%%%%%%%%%%%%%

\paragraph{}
The following page types may be useful, regardless of the kind of experiment you're running.

\subsubsection{login}

\paragraph{}
This is the first page that users see.  They enter their name here.  Names must be unique.  If someone has already attempted to start a test under that user name the user will not be able to proceed from the login page.

\paragraph{}
``login'' takes no arguments.

\subsubsection{audio\_test}

\paragraph{}
Some users have reported an inability to hear audio in LMEDS--perhaps there is an issue with the browser they are using and it won't load audio or their internet is too slow.  For this reason there is the audio\_test page.  This page can be presented right after a user logs in to save them time in the event that their browser is not correctly playing audio.

\paragraph{}
``audio\_test'' takes a single audio file as an argument.

e.g.

\begin{lstlisting}
audio_test example_audio_file
\end{lstlisting}

\paragraph{}
I recommend using a file that is similar to the stimuli they will hear in the experiment.  If the stimuli are long, a short test audio file might not represent the kind of load the user might encounter while doing the experiment.

\subsubsection{consent}

\paragraph{}
Presents the consent form.  If users opt not to consent, the test ends immediately.  If they consent, they proceed to the next page.

\paragraph{}
``consent'' takes one argument which specifies the consent form to display to users (the consent form is contained within the language dictionary)

e.g.

\begin{lstlisting}
consent main_consent
\end{lstlisting}

\subsubsection{text\_page}

\paragraph{}
A page that displays nothing but the text from a single text key.  This page could be used to provide task instructions to users, indicate that they should take a break, etc.

e.g.

\begin{lstlisting}
text_page first_block_instructions
\end{lstlisting}

\subsubsection{survey}

\paragraph{}
Specifies a survey page (see Section \ref{sec:surveySpec} for info on surveys).

\paragraph{}
``survey'' takes a single argument, the file name of the survey that it should load

e.g.

\begin{lstlisting}
survey presurvey
\end{lstlisting}

\paragraph{}
This would load the survey stored in the file called ``presurvey.txt''

\subsubsection{end}

\paragraph{}
The final page of the test.  

\paragraph{}
``end'' takes no arguments.

%%%%%%%%%%%%%%%%%%%%%
\subsection{Sequence file specification - experiment-specific pages}
\label{sec:sequenceSpecSpecific}
%%%%%%%%%%%%%%%%%%%%%

Here is a list of page types involving stimuli presentation.  

\subsubsection{prominence}

\paragraph{}
Users are presented with an audio file and the associated transcript.  They can click on a word to indicate that it is prominent.  This changes the selected word to \textbf{red}.  They can click it again to change it back to black.

\paragraph{}
``prominence'' takes the following arguments:

\begin{itemize}
\item name - the name of the audio file
\item transcriptName - the name of the text file
\item minPlays - the minimum number of times the audio file has to be played before the user can continue.
\item maxPlays - the maximum number of times the audio file can be played before the audio button is disabled. 
\item instructions - on the page users will encounter short (~one line) instructions reminding them of the task.  With this argument, you can present different stimuli with different short instructions (e.g. meaning, acoustics, vague).  \textbf{Default}: None (no specific instructions for this page--a generic instruction key .
\item presentAudio - either ``true'' or ``false'', specifies whether the audio is hidden or presented.  If ``false'' the values for ``minPlays'' and ``maxPlays'' are ignored. \textbf{Default}: True
\end{itemize}

e.g.

\begin{lstlisting}
prominence water water 1 3 acoustics true
prominence apples apples 1 1
\end{lstlisting}

\subsubsection{boundary}

\paragraph{}
Users are presented with an audio file and the associated transcript.  They can click on a word to mark the presence of a boundary after it.  This places a solid, vertical line after the word.  Clicking on the word again makes the vertical line disappear.

\paragraph{}
``boundary'' takes the same arguments as \textbf{prominence} pages but with the addition of one final, optional argument:

\paragraph{}
boundaryToken - specify the symbol to use for marking boundaries between words.  The default symbol used by LMDS is a vertical bar ``\texttt{|}'').

e.g.

\begin{lstlisting}
boundary water water 1 3 acoustics true &
boundary apples apples 1 2
\end{lstlisting}


\subsubsection{boundary\_and\_prominence}

\paragraph{}
A combination of prominences and boundaries.  Users first mark boundaries.  They then can mark prominences.  While marking prominences they can see but not change the boundaries that they marked--this is the only difference between this page and splitting the prominence and boundary task across two pages.

\paragraph{}
The arguments for boundary\_and\_prominence are the same as for boundary pages.

e.g.
\begin{lstlisting}
boundary_and_prominence water water 0 3 acoustics true
\end{lstlisting}

\subsubsection{audio\_list}

\paragraph{}
The user is presented with a single button.  On being pressed, a series of audio files will be played  On a follow up page, they could answer questions about the audio they heard.

\paragraph{}
audio\_list takes the following arguments:

\begin{itemize}
\item the length of pause between each audio file
\item the minimum number of times the audio series can be played
\item the maximum number of times the audio series can be played
\item the list of audio files to play, enclosed by \texttt{[ ]}
\end{itemize}

e.g.
\begin{lstlisting}
audio_list 1 1 1 [water apples water]
\end{lstlisting}

\subsubsection{audio\_choice}

\paragraph{}
This is used for presenting forced-choice-like tasks to the user with the number of audio files (stimuli) and responses set by the experimenter.

\paragraph{}
This page replaces the various functionality of the pages: same\_different\_stream, axb, ab, and their many variants.  These pages have been removed from LMEDS.

\paragraph{}
audio\_choice takes the following arguments:

\begin{itemize}
\item the short form instructions to present on the page
\item duration of pause for audio files
\item the minimum number of times the audio series can be played
\item the maximum number of times the audio series can be played
\item a list of lists of audio files (see discussion below)
\item a list of the labels for the response options
\item a list of names for labels for the audio files (should match the length of the audio files) \textbf{Default}: None (No labels are placed above the audio files--see discussion below)
\end{itemize}

e.g.
\begin{lstlisting}
audio_choice same_different_instr 0.5 1 -1 [[water apples]]
		[same different] [left_choice right_choice]
\end{lstlisting}

\paragraph{}
compare that example with the next one:

\begin{lstlisting}
audio_choice same_different_instr 0.5 1 -1 [[water] [apples]]
		[same different] [left_choice right_choice]
\end{lstlisting}

\paragraph{}
In the first example, with [[water apples]], there is one button that plays two audio files with a half second delay between each.  In the second example, with [[water] [apples]], there are two audio buttons and each is played only once.  Note that [water apples] will produce an error.  \textbf{The argument must be a list of lists}: [[water apples]] or [[water] [apples]].  Similarly with three arguments: [[water apples candy]] or [[water] [apples] [candy]] for the cases with either 1 button or 3 buttons, respectively.  In a minimal case, with only one audio file, you would write: [[audio\_name]], as in 

\begin{lstlisting}
audio_choice same_different_instr 0.5 1 -1 [audio_name]
		[p b] [audio_file]
\end{lstlisting}

\paragraph{}
Here the user would hear a single audio file and select option ``p'' or option ``b'' (such as in classical experiments investigating perception of VOT contrasts).

\paragraph{}
If you feel the response options are obvious--because there are two audio buttons and two corresponding response options--you could remove the last option like so:
\begin{lstlisting}
audio_choice same_different_instr 0.5 1 -1 [[water] [apples]]
		[same different]
\end{lstlisting}


%%%%%%%%%%%%%%%%%%%%%
\subsection{Parsing the LMEDS output}
%%%%%%%%%%%%%%%%%%%%%

\paragraph{}
The output to LMEDS is quite simple.  Each line has 4 components:
\begin{itemize}
\item	the page type
\item	the argument list
\item	peripheral information collected on the page (number of times users listened to audio files, time spent on the page, etc.)
\item	the users input in data-entry fields presented on the page
\end{itemize}

\paragraph{}
For most needs, the page type and user response are the most important bits of information.  The page type is separated from the rest of the row by the occurrence of the first comma.  The user input is separated from the rest of the row by the sequence ``;,''.  The argument list is enclosed in \texttt{[ ]} and separated from the peripheral information by a comma.  

\paragraph{}
The user response is a string of comma-separated values.  Every individual item in a data-entry field, except for textboxes, is represented by a 0 or 1, where 0 indicates the user did not choose that item and 1 indicates the user did choose that item.  So if the user has a choice between A or B and selects A, their output will look like 1, 0.  And if they select B: 0, 1.  For a boundary\_and\_prominence page, each word gets two digits in the output (one for boundary and one for prominence).

\paragraph{}
Let's look at a real example.  Here we have the sequence:

\begin{lstlisting}
boundary_and_prominence water water 0 3 acoustics true
same_different water water 1 -1
axb water water water 0 2
ab water 0 2
\end{lstlisting}

\paragraph{}
And here is one possible output for it

\begin{lstlisting}
boundary_and_prominence,[water, water, 0, -1, acoustics, true],
	1,1,0:8.9;,0,1,0,0,0,0,0,0,1,0,0,0,0,0
same_different,[water, water, 1, -1],0,0,0:4.8;,1,0
axb,[water, water, water, 0, 2],0,0,0:6.7;,0,1
ab,[water, 0, 2],0,0,0:3.3;,1,0
\end{lstlisting}

\paragraph{}
If we separate out just the page name and the user response we can easily see the data that we want to analyze: 

\begin{lstlisting}
boundary_and_prominence;,0,1,0,0,0,0,0,0,1,0,0,0,0,0
same_different;,1,0
axb;,0,1
ab;,1,0
\end{lstlisting}

\paragraph{}
LMEDS offers some useful post-processing that can help prep data for analysis (see \ref{users_scripts}.  Otherwise, with the above information, you should have everything you need to do your own analysis.

%%%%%%%%%%%%%%%%%%%%%
\subsection{CGI template}
\label{sec:cgitemplate}
%%%%%%%%%%%%%%%%%%%%%

\paragraph{}
The cgi file  specifies the location of the experiment files, the sequence name, and the dictionary name.  Using these pieces of information, LMEDS will correctly locate the inputs and place the outputs.  The CGI file itself is quite short.  You can copy and paste the text below, changing only the four arguments to the function \textbf{runExperiment}.

\paragraph{}

\begin{tcolorbox}[breakable,colback=white,colframe=blue,width=\dimexpr\textwidth+12mm\relax,enlarge left by=-6mm,enlarge right by=6mm]

\begin{lstlisting}
\#!/usr/bin/env python
\# -*- coding: utf-8 -*-

import experiment_runner
experiment_runner.runExperiment(`name_of_folder_in_tests', 
				`name_of_sequence_file.txt', 
				`name_of_dictionary_file.txt', 
				disableRefresh=True)
\end{lstlisting}

\end{tcolorbox}

