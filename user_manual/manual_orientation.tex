% !TEX root = LMEDS_manual.tex

%%%%%%%%%%%%%%%%%%%%%
\section{Orientation}
%%%%%%%%%%%%%%%%%%%%%

LMEDS comes packaged with the following folders:

\begin{itemize}
\item cgi-bin

Small code snippets (with the extension .cgi) go here--one for each experiment

\item html

Contains static HTML, Javascript, and CSS files

\item imgs

Image files used by LMEDS

\item lmeds

The main repository for code.

\item lmeds/user\_scripts

Within the lmeds code directory are some scripts that users can use in setting up their experiments and in post-processing their data.

\item tests

Holds the resource files for each experiment

\item user\_manual

Contains the user manual

\end{itemize}

\paragraph{}

If you are creating an experiment, you'll need to add one file to cgi-bin and you'll need to add a folder to tests, containing all of the resource files used in the experiment--the other folders can all be ignored. The next section goes into the details of how to create the 

\subsection{Getting started with LMEDS}

This manual describes how to create files used by LMEDS from scratch.  You might find it useful to reference the experiment files that come with LMEDS, located in /tests/lmeds\_demo.  In it, you'll find an example dictionary (english.txt), sequence file (sequence.txt), input files contained in the (audio/ and txt/ folders), survey files (presurvey.txt and postsurvey.txt), and some sample output files (stored in the folder output/).
\textbf{When you are starting a new experiment from scratch, you might find it useful to start with the demo files provided.}

\begin{tcolorbox}[breakable,colback=white,colframe=red,width=\dimexpr\textwidth+12mm\relax,enlarge left by=-6mm]


