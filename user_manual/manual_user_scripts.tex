% !TEX root = LMEDS_manual.tex

%%%%%%%%%%%%%%%%%%%%%
\section{User scripts}
\label{sec:users_scripts}
%%%%%%%%%%%%%%%%%%%%%

\paragraph{}
Everything in LMEDS can be done by hand, but the provided scripts located in \\\texttt{/lmeds/user\_scripts} enable you to easily perform certain routine tasks.

\paragraph{}
All scripts can be run in one of three ways.  

\paragraph{}
\textbf{The first}, and most user-friendly option, is from the web browser.  Begin by entering the URL of the relevant experiment.  Add a `?' and then the script name.  For example,
\begin{lstlisting}
http://127.0.0.1:8123/cgi-bin/lmeds_demo.py?sequence_check
\end{lstlisting}

\paragraph{}
Each script requires certain information.  Conveniently, all of this information is already specified in the .cgi/.py file, so if you choose this option, you don't need to enter any further information--unlike the other two options.  The output is displayed directly in the browser--except for get\_duration and post\_process\_results which may display confidential information (e.g. if the subjects write their full name as their user id).  Instead, for those, a log file is generated that the experimenter can access later.

\paragraph{}
In order to run scripts via the web browser, the cgi file must be configured properly.  See section \ref{sec:cgitemplate}.

\paragraph{}
\textbf{Second}, the user scripts can be run from the command line (\textit{cmd} on Windows or \textit{terminal} on OS X or Linux).  \texttt{python <<script\_name>>.py -h} will print out the options for that script.  For example, 
\begin{lstlisting}
python sequence_check.py -h
\end{lstlisting}

\paragraph{}
\textbf{Third}, user scripts can be run from within a python development environment, such as IDLE, which is bundled with every version of python.  If you open IDLE, choose \texttt{File >> Open} and open the desired script.  The script will open in a new window.  Select \texttt{Run >> Run Module}.  The application will launch.  It will ask if you want to go into interactive mode.  Type ``yes".  The script will then ask question-by-question for all of the information it needs to run.


%%%%%%%%%%%%%%%%
\subsection{Generating the language dictionary}

\paragraph{}
Given a sequence file, python can generate an empty dictionary that contains all of the keys needed by the pages used in the sequence file.  The script can also update existing dictionaries used in other experiments or in cases where pages have been added or removed from a sequence file.

\begin{lstlisting}
python generate_language_dictionary.py -m update lmeds_demo 
				sequence.txt english.txt
\end{lstlisting}

\paragraph{}
There are three versions of this script that can be run from the web browser.

\paragraph{}
To create a dictionary for a new project:
\begin{lstlisting}
http://127.0.0.1:8123/cgi-bin/lmeds_demo.py?create_dictionary
\end{lstlisting}

\paragraph{}
To update an existing dictionary (preserves existing keys and text):
\begin{lstlisting}
http://127.0.0.1:8123/cgi-bin/lmeds_demo.py?update_dictionary
\end{lstlisting}

\paragraph{}
To remove keys that are no longer relevant to a sequence file.  Note that if there is a language dictionary
that is common to two different sequence files, running crop in this manner on one sequence file may
make the dictionary no longer work with the other sequence.
\begin{lstlisting}
http://127.0.0.1:8123/cgi-bin/lmeds_demo.py?crop_dictionary
\end{lstlisting}


%%%%%%%%%%%%%%%%
\subsection{Verifying experiment integrity}

\paragraph{}
This user script ensures that an experiment is ready to be run.  It makes sure that all text keys are in the dictionary, and that LMEDS can access all the wav and text resources that are included in the sequence file.  \textbf{Even if you use this script, you'll still want to run through the experiment at least once before you start collecting data.}  However, this script will save you some headaches if you've misspelled resource names in the sequence file, for example.

\begin{lstlisting}
python sequence_check.py lmeds_demo sequence.txt english.txt true
\end{lstlisting}

\paragraph{}
To run this from a web browser:
\begin{lstlisting}
http://127.0.0.1:8123/cgi-bin/lmeds_demo.py?sequence_check
\end{lstlisting}


%%%%%%%%%%%%%%%%
\subsection{Getting test duration}

\paragraph{}
This simple script will output the length of time each user spent on your experiment, along with the average time and the standard deviation.

\begin{lstlisting}
python get_test_duration.py lmeds_demo sequence.txt
\end{lstlisting}

\paragraph{}
To run this from a web browser:
\begin{lstlisting}
http://127.0.0.1:8123/cgi-bin/lmeds_demo.py?get_test_duration
\end{lstlisting}


%%%%%%%%%%%%%%%%
\subsection{Post-processing the results}
\label{postprocessresults}

\paragraph{}
Each user will have their data stored in a separate file.  For doing many kinds of analysis, this is not convenient.  Furthermore, the questions are removed from the responses and the responses are mixed with different kinds of questions.

\paragraph{}
This script was made to remedy some of these issues.  First, items are separated by page type.  Then, for survey items and rpt experiments, items are transposed, paired with the inputs (survey questions and transcripts, respectively) and then all of the data is combined into one spreadsheet for each page type.  This makes it much more convenient to explore the data and to do statistical analysis.

\paragraph{}
Currently, LMEDS may experience duplicate data entries--in particular when the \textbf{lmeds\_main.py}'s \textit{disableRefreshFlag} is set to False and the user hits back or refresh.  The code will warn when it detects multiple items and prevent the user from continuing but there is also the option to remove duplicate entries when they appear (printing the found instances).  This happens before anything else in the script.

\paragraph{}
\begin{tcolorbox}[colback=white,colframe=red,width=\dimexpr\textwidth+12mm\relax,enlarge left by=-6mm,enlarge right by=6mm]
The code to remove duplicate entries is a little naive.  If there is a situation where two identical lines could appear adjacent in your sequence file, LMEDS will assume in the output that the two lines are duplicates of each other and automatically remove the second one if \textit{removeDuplicatesFlag} is set to True.  If set to False, it makes no assumptions about the relationships between lines.  True is the convenient option, False is the safe option--if you are concerned.
\end{tcolorbox}

\paragraph{}
The script can also remove any items used in the experiment that the user does not want to include in the output--for example test items or items that are removed from the study before data collection concludes.  By default the post processing user script doesn't remove any data.

\paragraph{}
\textbf{An important note on randomized sequences}, if a sequence is flagged as `randomized' in it's cgi file, the order each participant did the stimuli in will be randomized.  The post-processing code will recognize this automatically arrange all user responses in the order specified in the main sequence file.  However, the post processing script will also output the order that each item was presented in.  This can be seen in the example output files included in the following folder with LMEDS

/LMEDS/tests/lmeds\_demo/output/LMEDS\_Randomized\_Demo/duplicates\_removed\_results/

\paragraph{}
Output data can be found in subfolders created in the output folder of the experiment.

\begin{lstlisting}
python post_process_results.py testName sequenceFN
	removeDuplicatesFlag removeItemList=[]
python post_process_results.py lmeds_demo sequence.txt false
\end{lstlisting}

\paragraph{}
To run this from a web browser (assumes \textit{removeDuplicatesFlag} is True):
\begin{lstlisting}
http://127.0.0.1:8123/cgi-bin/lmeds_demo.py?post_process_results
\end{lstlisting}

